\documentclass[12pt,a4j]{jarticle}
\usepackage{graphicx}
\begin{document}
\title{基礎プログラミングおよび演習 レポート #15}
\author{学籍番号, 氏名 (ペア: 氏名・学籍番号または「個人作業」)}
\date{提出日付}
\maketitle

\section{構想・計画・設計}

(どのような構想で絵を生成したか、プログラムはどう設計したか)

\section{プログラムコード}

\begin{verbatim}
(ここにプログラムのソースコードを入れる)
\end{verbatim}

\section{プログラムの説明}

(プログラムのどの部分が何をしているかを説明する)

\section{生成された動画の説明}

(画像を入れたい場合は下記で。mypicture.psというファイル名は適宜変更)
\begin{center}
\includegraphics[width=12cm]{mypicture.ps}
\end{center}

(どのような動画という説明を書く。)
(動画ファイルはアップロードで提出。プログラムコードと動画が一致していること。)

\section{開発過程の説明}

(誰が何を分担し、どのような過程を経てプログラムが完成したか。各作業の日時と担当者の記録があるとよい。)

\section{考察}

(考察は必須かつ重要。課題をやって分かったことや感想など。)

\section{アンケート}

\subsection{Q1:うまく分担して課題プログラムを開発できましたか。}

(ここにQ1の回答を記入)

\subsection{Q2:複数で分担する際に注意すべきことは何だと思いましたか。}

(ここにQ2の回答を記入)

\subsection{Q3:ここまでの科目全体を通して、学べたこと、学びたかったけど学べなかったことは何ですか。その他感想や、この科目の今後改善した方がよいこと、今後も維持したことがよいこ との指摘もどうぞ。}

(ここにQ3の回答を記入)

\end{document}
